% !TeX root = ../main.tex

\documentclass[../main.tex]{subfiles}

\begin{document}
\section{Model Parametrization}
\begin{definition}\label{M:Parameter}
    For any point matrix $\tensor{X^\pm_{k,n}\left(\theta,\phi_1,\phi_2,\dots,\phi_n\right)}{}$,
    $n$-dimensional vector $\tensor{\theta}{}$
    is defined as \textit{position parameter}.
\end{definition}
\begin{definition}\label{M:Vector}
    For point matrix $\tensor{X}{}$,
    $\left(n+1\right)$-dimensional column vector $\tensor{p}{}\defeq \frac{1}{k} \tensor{X}{}\cdot\tensor{e}{^1}=\frac{1}{k} \tensor{X}{_1}$
    is defined as \textit{position vector}.
\end{definition}
\begin{definition}\label{M:Vector:Set}
    $P^\ast\left(k,n\right)$ is a set of position vectors.
\end{definition}
\begin{lemma}\label{M:Vector:Position}
    For point matrix $X=PO$
    where $P$ and $O$ are position and orientation matrix respectively,
    $\tensor{p}{}=\frac{1}{k}\tensor{X}{_1}=\frac{1}{k}\tensor{P}{_1}$.
\end{lemma}
\begin{proof}[\proofof{M:Vector:Position}]
    \begin{align*}
        \tensor{p}{^i}
         & = \frac{1}{k}\tensor{X}{^i_1}                         &  & \text{\cref{M:Vector}}       \\
         & = \frac{1}{k}\sum_j{\tensor{P}{^i_j}\tensor{O}{^j_1}} &  & \text{\cref{Matrix:Product}} \\
         & = \frac{1}{k}\tensor{P}{^i_1}                         &  & \text{\cref{M:Orientation}}  \\
        \tensor{p}{}                                                                               \\
         & = \frac{1}{k}\tensor{P}{_1}                           &  & \qedhere
    \end{align*}
\end{proof}
\begin{lemma}\label{M:Vector:Value}
    Given position parameter $\tensor{\theta}{}$, position vector can be evaluated as the following.
    \begin{equation*}
        \psi_0^{-1}: \tensor{\theta}{} \mapsto \tensor{p}{}
        = \frac{1}{k}
        \begin{pmatrix}
            \prod_{j\in\Set{1..n}}{\cos_k{\tensor{\theta}{^j}}}                                \\
            \sin_k{\tensor{\theta}{^{i-1}}}\prod_{j\in\Set{i..n}}{\cos_k{\tensor{\theta}{^j}}} \\
            \sin_k{{\tensor{\theta}{^n}}}                                                      \\
        \end{pmatrix}\text{.}
    \end{equation*}
\end{lemma}
\begin{proof}[\proofof{M:Vector:Value}]
    Simplify \cref{M:Vector:Position,M:Position}.
\end{proof}
\begin{lemma}\label{M:Parameter:Value}
    Given position vector $\tensor{p}{}$, position parameter can be calculated as the following.
    \begin{align*}
        \psi_0
         & : \tensor{p}{} \mapsto \tensor{\theta}{}
        =
        \begin{pmatrix}
            \arcsin_k^{\sign{\tensor{p}{^1}}}{\frac{k\tensor{p}{^2}}{\prod_{j\in\Set{2..n}}{\cos_k{\tensor{\theta}{^j}}}}} \\
            \arcsin_k{\frac{k\tensor{p}{^{i+1}}}{\prod_{j\in\Set{i+1..n}}{\cos_k{\tensor{\theta}{^j}}}}}                   \\
            \arcsin_k{k\tensor{p}{^{n+1}}}                                                                                 \\
        \end{pmatrix} \\
         & \in
        \begin{cases}
            P \to \left(-\frac{\pi}{k}, \frac{\pi}{k}\right]\times\left[-\frac{1}{2}\frac{\pi}{k}, \frac{1}{2}\frac{\pi}{k}\right]^{n-1} & \text{if $k>0$}   \\
            P \to \R^{n}                                                                                                                 & \text{if $k\le0$} \\
        \end{cases}
    \end{align*}
    where $\cos_k\left(\arcsin_k^{\pm}\left(x\right)\right) = \pm \cos_k\left(\arcsin_k\left(x\right)\right)$.
\end{lemma}
\begin{proof}[\proofof{M:Parameter:Value}]
    From \cref{M:Vector:Value},
    \begin{align*}
        kp                                                                                 & = \begin{pmatrix}
                                                                                                   \prod_{j\in\Set{1..n}}{\cos_k{\tensor{\theta}{^j}}}                                \\
                                                                                                   \sin_k{\tensor{\theta}{^{i-1}}}\prod_{j\in\Set{i..n}}{\cos_k{\tensor{\theta}{^j}}} \\
                                                                                                   \sin_k{{\tensor{\theta}{^n}}}                                                      \\
                                                                                               \end{pmatrix}                \\
        \sin_k{{\tensor{\theta}{^n}}}                                                      & = k\tensor{\theta}{^{n+1}}                                                                                            \\
        \tensor{\theta}{^n}                                                                & = \arcsin_k{k\tensor{\theta}{^{n+1}}}                                                                                 \\
        \sin_k{\tensor{\theta}{^{i-1}}}\prod_{j\in\Set{i..n}}{\cos_k{\tensor{\theta}{^j}}} & = k\tensor{\theta}{^{i}}                                                                                              \\
        \sin_k{\tensor{\theta}{^{i-1}}}                                                    & = \frac{k\tensor{\theta}{^{i}}}{\prod_{j\in\Set{i..n}}{\cos_k{\tensor{\theta}{^j}}}}                                  \\
        \tensor{\theta}{^{i-1}}                                                            & = \arcsin_k{\frac{k\tensor{\theta}{^{i}}}{\prod_{j\in\Set{i..n}}{\cos_k{\tensor{\theta}{^j}}}}}                       \\
        \tensor{\theta}{^i}                                                                & = \arcsin_k{\frac{k\tensor{\theta}{^{i+1}}}{\prod_{j\in\Set{i+1..n}}{\cos_k{\tensor{\theta}{^j}}}}}                   \\
        \tensor{p}{^1}                                                                     & = \prod_{j\in\Set{1..n}}{\cos_k{\tensor{\theta}{^j}}}                                                                 \\
        \tensor{p}{^1}                                                                     & = \cos_k{\tensor{\theta}{^1}}\prod_{j\in\Set{2..n}}{\cos_k{\tensor{\theta}{^j}}}                                      \\
        \sign{\tensor{p}{^1}}                                                              & = \sign{\cos_k{\tensor{\theta}{^1}}}\prod_{j\in\Set{2..n}}{\sign{\cos_k{\tensor{\theta}{^j}}}}                        \\
        \sign{\tensor{p}{^1}}                                                              & = \sign{\cos_k{\tensor{\theta}{^1}}}\prod_{j\in\Set{2..n}}{1}                                                         \\
        \sign{\tensor{p}{^1}}                                                              & = \sign{\cos_k{\tensor{\theta}{^1}}}                                                                                  \\
        \theta                                                                             & = \begin{pmatrix}
                                                                                                   \arcsin_k^{\sign{\tensor{p}{^1}}}{\frac{k\tensor{p}{^2}}{\prod_{j\in\Set{2..n}}{\cos_k{\tensor{\theta}{^j}}}}} \\
                                                                                                   \arcsin_k{\frac{k\tensor{p}{^{i+1}}}{\prod_{j\in\Set{i+1..n}}{\cos_k{\tensor{\theta}{^j}}}}}                   \\
                                                                                                   \arcsin_k{k\tensor{p}{^{n+1}}}                                                                                 \\
                                                                                               \end{pmatrix}
    \end{align*}
\end{proof}
\begin{lemma}\label{M:CoordinateChart}
    \begin{equation*}
        \Psi=\set{\psi|
            \psi^{-1}
            \in S^n \to P
            :\tensor{\theta}{}\mapsto P_{k,n}\left(\tensor{\theta}{}+\tensor{x}{}\right)
            \text{ for }
            \tensor{x}{} \in \R^n
        }
    \end{equation*} is a coordinate chart of a $C^\infty$ differential structure on $P$
    for \begin{equation*}
        S=
        \begin{cases}
            \left(-\frac{1}{2}\frac{\pi}{k},+\frac{1}{2}\frac{\pi}{k}\right) & \text{$k>0$}   \\
            \R                                                               & \text{$k\le0$}
        \end{cases}
    \end{equation*}
\end{lemma}
\begin{proof}[\proofof{M:CoordinateChart}]
    From \cref{Manifold}, It is sufficient to shows that
    \begin{APAenumerate}
        \item $R_\psi$ is an open subset of real vector space (defined),
        \item $\bigcup_{\psi\in\Psi} D_\psi = P$ (obvious),
        \item transition map is in differentability class $C^\infty$.
    \end{APAenumerate}
    \begin{subproof}{$\bigcup_{\psi\in\Psi} D_\psi = P$}
        \begin{align*}
            M \in P
                                         & \implies \exists \theta_0, M=P_{k,n}\left(\theta_0\right)                  \\
                                         & \implies \exists \theta_0, M=P_{k,n}\left(0 + \theta_0\right)              \\
                                         & \implies M\in R_{\psi^{-1}}                                                \\
                                         & \implies M\in D_{\psi}                                                     \\
                                         & \implies M\in \bigcup_{\psi\in\Psi} D_\psi                                 \\
            P                            & \subset\bigcup_{\psi\in\Psi}D_\psi                                         \\
            M\in \bigcup_{\psi\in\Psi} D_\psi
                                         & \implies \exists \psi\in\Psi, M\in D_\psi                                  \\
                                         & \implies \exists \psi\in\Psi, M\in R_{\psi^{-1}}                           \\
                                         & \implies \exists x_0\exists\theta\in S^n, M=P_{k,n}\left(\theta+x_0\right) \\
                                         & \implies \exists x_0, M=P_{k,n}\left(0+x_0\right)                          \\
                                         & \implies \exists x_0, M=P_{k,n}\left(x_0\right)                            \\
                                         & \implies M \in P                                                           \\
            \bigcup_{\psi\in\Psi}D_\psi  & \subset P                                                                  \\
            \bigcup_{\psi\in\Psi} D_\psi & = P
        \end{align*}
    \end{subproof}
    \begin{subproof}{every transition map is in differentability class $C^\infty$}
        Consider $\psi_1, \psi_2 \in \Psi$ and $x_1, x_2 \in \R^n$ where
        \begin{equation*}
            \psi_i^{-1}
            \in S^n \to R
            :\tensor{\theta}{}\mapsto P\left(\tensor{\theta}{}+x_i\right)
            \text{.}
        \end{equation*}
        If $\psi_1^{-1}\left(\theta_1\right)=\psi_2^{-1}\left(\theta_2\right)$,
        let $\phi_i = \theta_i+x_i$.
        \begin{align*}
            \frac{1}{k}
            \begin{pmatrix}
                \prod_{j\in\Set{1..n}}{\cos_k{\tensor{{\phi_1}}{^j}}}                                  \\
                \sin_k{\tensor{{\phi_1}}{^{i-1}}}\prod_{j\in\Set{i..n}}{\cos_k{\tensor{{\phi_1}}{^j}}} \\
                \sin_k{{\tensor{{\phi_1}}{^n}}}                                                        \\
            \end{pmatrix}
                                                                                                                      & =
            \frac{1}{k}
            \begin{pmatrix}
                \prod_{j\in\Set{1..n}}{\cos_k{\tensor{{\phi_2}}{^j}}}                                  \\
                \sin_k{\tensor{{\phi_2}}{^{i-1}}}\prod_{j\in\Set{i..n}}{\cos_k{\tensor{{\phi_2}}{^j}}} \\
                \sin_k{{\tensor{{\phi_2}}{^n}}}                                                        \\
            \end{pmatrix} & \text{\cref{M:Vector:Value}}        \\
            \begin{pmatrix}
                \prod_{j\in\Set{1..n}}{\cos_k{\tensor{{\phi_1}}{^j}}}                                  \\
                \sin_k{\tensor{{\phi_1}}{^{i-1}}}\prod_{j\in\Set{i..n}}{\cos_k{\tensor{{\phi_1}}{^j}}} \\
                \sin_k{{\tensor{{\phi_1}}{^n}}}                                                        \\
            \end{pmatrix}
                                                                                                                      & =
            \begin{pmatrix}
                \prod_{j\in\Set{1..n}}{\cos_k{\tensor{{\phi_2}}{^j}}}                                  \\
                \sin_k{\tensor{{\phi_2}}{^{i-1}}}\prod_{j\in\Set{i..n}}{\cos_k{\tensor{{\phi_2}}{^j}}} \\
                \sin_k{{\tensor{{\phi_2}}{^n}}}                                                        \\
            \end{pmatrix}
        \end{align*}

        If $k\ge 0$, by mathematical induction, $\phi_1 = m\frac{\pi}{k}\pm\phi_2$.

        Otherwise, by mathematical induction, $\phi_1 = \phi_2$.

        Hence, the transition map $\tau_{1,2} = \psi_2 \circ \psi_1^{-1}$
        is in the form of $\theta \mapsto c\pm\theta$
        and is in differentability class $C^\infty$.
    \end{subproof}
\end{proof}
\subsection{Locus of position vector}
\begin{lemma}\label{SphericalLocus}
    For $k>0$, $P^\ast$ is a $\left(n+1\right)$-sphere of radius $k^{-1}$.
\end{lemma}
\begin{proof}[\proofof{SphericalLocus}]
    Simplify \cref{M:Vector:Value} using \cref{M:Trigonometry:Pythagorean}.
\end{proof}
\begin{lemma}\label{HyperbolicLocus}
    For $k<0$, $P^\ast$ is a forward sheet of a two-sheeted $\left(n+1\right)$-hyperboloid of radius $k^{-1}$.
\end{lemma}
\begin{proof}[\proofof{HyperbolicLocus}]
    Simplify \cref{M:Vector:Value} using \cref{M:Trigonometry:Pythagorean}.
\end{proof}
\begin{lemma}\label{EuclideanLocus}
    For $k\to0$, $P^\ast$ is a $n$-Euclidean manifold at infinity.
\end{lemma}
\begin{proof}[\proofof{EuclideanLocus}]
    Using limits.
\end{proof}

\end{document}