% !TeX root = ../main.tex
\documentclass[../main.tex]{subfiles}
\begin{document}
\begin{figure}
    \centering
    \includestandalone{sections/figures/2dplot}
    \caption{Generalized trigonometric functions as function of \(k\)}\label{TrigonometryPlotted}
    \figurenote{This graph shows the value of generalized trigonometric functions as solid line and trigonometric and hyperbolic functions in the unused domain as dashed line.}
\end{figure}
\begin{figure}
    \centering
    \includestandalone{sections/figures/sine}
    \caption{Generalized sine function}\label{TrigonometrySinePlotted}
\end{figure}
\begin{figure}
    \centering
    \includestandalone{sections/figures/sine_}
    \caption{Generalized sine function variant}\label{TrigonometrySineVarPlotted}
\end{figure}
\begin{figure}
    \centering
    \includestandalone{sections/figures/cosine}
    \caption{Generalized cosine function}\label{TrigonometryCosinePlotted}
\end{figure}
\begin{figure}
    \centering
    \includestandalone{sections/figures/tangent}
    \caption{Generalized tangent function}\label{TrigonometryTangentPlotted}
\end{figure}
\begin{figure}
    \centering
    \includestandalone{sections/figures/tangent_}
    \caption{Generalized tangent function variant}\label{TrigonometryTangentVarPlotted}
\end{figure}
\begin{figure}
    \centering
    \includestandalone{sections/figures/pointmapping}
    \caption{Relation of representations}\label{PointMapping}
\end{figure}
\end{document}