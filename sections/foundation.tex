% !TeX root = ../main.tex

\documentclass[../main.tex]{subfiles}

\begin{document}
\section{Model Foundation}
\subsection{Trigonometry}
\begin{definition}\label{M:Trigonometry}
    \textit{Generalized trigonometric functions} $f_k:\R\to\R$ and $f_k^\ast:\R\to\R$ are defined as
    \begin{align*}
        f_k\left(\theta\right)      & \defeq
        \begin{cases}
            g\left(k\theta\right) & \text{if $k\geq0$,} \\
            h\left(k\theta\right) & \text{otherwise,}   \\
        \end{cases} \\
        f_k^\ast\left(\theta\right) & \defeq
        \begin{cases}
            g\left(k\theta\right)  & \text{if $k\geq0$,} \\
            h\left(-k\theta\right) & \text{otherwise,}   \\
        \end{cases}
    \end{align*}
    where $g$ (resp. $h$) are the associated trigonometric (resp. hyperbolic) function.
\end{definition}
\begin{example}[Generalized sine functions]\label{M:Trigonometry:Sine}
    \begin{align*}
        \sin_k{\theta}      & \defeq
        \begin{cases}
            \sin\left(k\theta\right)  & \text{if $k\geq0$,} \\
            \sinh\left(k\theta\right) & \text{otherwise.}   \\
        \end{cases}                               \\
        \sin_k^\ast{\theta} & \defeq
        \begin{cases}
            \sin\left(k\theta\right)   & \text{if $k\geq0$,} \\
            \sinh\left(-k\theta\right) & \text{otherwise.}   \\
        \end{cases}                              \\
                            & =
        \begin{cases}
            \sin\left(k\theta\right)   & \text{if $k\geq0$,} \\
            -\sinh\left(k\theta\right) & \text{otherwise.}   \\
        \end{cases}                              \\
                            & = \begin{cases}
                                    \sin\left(\abs{k}\theta\right)  & \text{if $k\geq0$,} \\
                                    \sinh\left(\abs{k}\theta\right) & \text{otherwise.}   \\
                                \end{cases}
    \end{align*}
    (see \cref{TrigonometrySinePlotted})
\end{example}
\begin{example}[Generalized cosine functions]\label{M:Trigonometry:Cosine}
    \begin{align*}
        \cos_k{\theta}      & \defeq
        \begin{cases}
            \cos\left(k\theta\right)  & \text{if $k\geq0$,} \\
            \cosh\left(k\theta\right) & \text{otherwise.}   \\
        \end{cases}                         \\
        \cos_k^\ast{\theta} & \defeq
        \begin{cases}
            \cos\left(k\theta\right)   & \text{if $k\geq0$,} \\
            \cosh\left(-k\theta\right) & \text{otherwise.}   \\
        \end{cases}                        \\
                            & = \begin{cases}
                                    \cos\left(k\theta\right)  & \text{if $k\geq0$,} \\
                                    \cosh\left(k\theta\right) & \text{otherwise.}   \\
                                \end{cases} \\
                            & = \cos_k{\theta}
    \end{align*}
    (see \cref{TrigonometryCosinePlotted})
\end{example}
\begin{theorem}[Pythagorean's identity equivalence]\label{M:Trigonometry:Pythagorean}
    \begin{equation*}
        \cos_k^2{\theta} +\sign{k}\sin_k^2{\theta} = 1
    \end{equation*}
\end{theorem}
\begin{proof}[\proofof{M:Trigonometry:Pythagorean}]
    Proof by exhaustion.
\end{proof}
\begin{proposition}[Generalized trigonometric functions of sum of arguments]\label{M:Trigonometry:Sum}
    \begin{align*}
        \sin_k\left(\theta+\phi\right)
         & = \sin_k{\theta}\cos_k{\phi}+\cos_k{\theta}\sin_k{\phi}           \\
        \sin_k^\ast\left(\theta+\phi\right)
         & = \sin_k^\ast{\theta}\cos_k{\phi}+\cos_k{\theta}\sin_k^\ast{\phi} \\
        \cos_k\left(\theta+\phi\right)
         & = \cos_k{\theta}\cos_k{\phi}-\sign{k}\sin_k{\theta}\sin_k{\phi}   \\
         & = \cos_k{\theta}\cos_k{\phi}-\sin_k^\ast{\theta}\sin_k{\phi}      \\
         & = \cos_k{\theta}\cos_k{\phi}-\sin_k{\theta}\sin_k^\ast{\phi}
    \end{align*}
\end{proposition}
\begin{proof}[\proofof{M:Trigonometry:Sum}]
    Proof by exhaustion.
\end{proof}
\begin{proposition}[Derivative of generalized trigonometric functions]\label{M:Trigonometry:Derivative}
    \begin{align*}
        {\sin_k}^\prime{\theta}      & = k\cos_k{\theta}       \\
        {\sin_k^\ast}^\prime{\theta} & = \abs{k}\cos_k{\theta} \\
        {\cos_k}^\prime{\theta}      & = -k\sin_k^\ast{\theta}
    \end{align*}
\end{proposition}
\begin{proof}[\proofof{M:Trigonometry:Derivative}]
    Proof by exhaustion.
\end{proof}
\subsection{Matrices}
\begin{definition}\label{M:Rotation}
    \textit{Generalized rotation matrix} is defined as
    \begin{align*}
        R_k\left(\theta\right) & \defeq
        \begin{bmatrix}
            \cos_k{\theta} & -\sin_k^\ast{\theta} \\
            \sin_k{\theta} & \cos_k{\theta}       \\
        \end{bmatrix}\text{,}
    \end{align*}
    where $\theta\in\R$.
\end{definition}
\begin{corollary}[Generalized rotation matrix at zero]\label{M:Rotation:Identity}
    \begin{equation*}
        R_k\left(0\right)
        =
        I_2
    \end{equation*}
\end{corollary}
\begin{proof}[\proofof{M:Rotation:Identity}]
    Obvious
\end{proof}
\begin{corollary}[Generalized rotation matrix of sum of arguments]\label{M:Rotation:Sum}
    \begin{equation*}
        R_k\left(\theta\right)R_k\left(\phi\right)=R_k\left(\theta+\phi\right)
    \end{equation*}
\end{corollary}
\begin{proof}[\proofof{M:Rotation:Sum}]
    \begin{align*}
        R_k\left(\theta\right)R_k\left(\phi\right)
         & =\begin{bmatrix}
                \cos_k{\theta} & -\sin_k^\ast{\theta} \\
                \sin_k{\theta} & \cos_k{\theta}       \\
            \end{bmatrix}\begin{bmatrix}
                             \cos_k{\phi} & -\sin_k^\ast{\phi} \\
                             \sin_k{\phi} & \cos_k{\phi}       \\
                         \end{bmatrix}                                                       & \text{(\cref{M:Rotation})}              \\
         & =\begin{bmatrix}
                \cos_k{\theta}\cos_k{\phi}+\left(-\sin_k^\ast{\theta}\right)\sin_k{\phi} &
                \cos_k{\theta}\left(-\sin_k^\ast{\phi}\right)+\left(-\sin_k^\ast{\theta}\right)\cos_k{\phi} \\
                \sin_k{\theta}\cos_k{\phi}+\cos_k{\theta}\sin_k{\phi}                    &
                \sin_k{\theta}\left(-\sin_k^\ast{\phi}\right)+\cos_k{\theta}\cos_k{\phi}                    \\
            \end{bmatrix} & \text{(\cref{Matrix:Product})}                                \\
         & =\begin{bmatrix}
                \cos_k{\theta}\cos_k{\phi}-\sin_k^\ast{\theta}\sin_k{\phi} &
                -\left(\sin_k^\ast{\theta}\cos_k{\phi}+\cos_k{\theta}\sin_k^\ast{\phi}\right) \\
                \sin_k{\theta}\cos_k{\phi}+\cos_k{\theta}\sin_k{\phi}      &
                \cos_k{\theta}\cos_k{\phi}-\sin_k{\theta}\sin_k^\ast{\phi}                    \\
            \end{bmatrix}               & \text{(simplify)}                                              \\
         & =\begin{bmatrix}
                \cos_k{\theta+\phi} & -\sin_k^\ast{\theta+\phi} \\
                \sin_k{\theta+\phi} & \cos_k{\theta+\phi}       \\
            \end{bmatrix}                                             & \text{(\cref{M:Trigonometry:Sum})}                             \\
         & = R_k\left(\theta+\phi\right)                                                                  & \text{(\cref{M:Rotation})} \\
        R_k\left(\theta\right)R_k\left(\phi\right)
         & = R_k\left(\theta+\phi\right)                                                                  & \qedhere
    \end{align*}
\end{proof}
\begin{corollary}[Inverse of generalized rotation matrix]\label{M:Rotation:Inverse}
    \begin{equation*}
        R_k\left(\theta\right)^{-1} = R_k\left(-\theta\right)
    \end{equation*}
\end{corollary}
\begin{proof}[\proofof{M:Rotation:Inverse}]
    \begin{align*}
        R_k\left(\theta\right)R_k\left(-\theta\right)
         & = R_k\left(0\right) & \text{(\cref{M:Rotation:Sum})}      \\
         & = I_2               & \text{(\cref{M:Rotation:Identity})} \\
        R_k\left(-\theta\right)R_k\left(\theta\right)
         & = R_k\left(0\right) & \text{(\cref{M:Rotation:Sum})}      \\
         & = I_2               & \text{(\cref{M:Rotation:Identity})}
    \end{align*}
    \begin{equation*}
        R_k\left(\theta\right)R_k\left(-\theta\right)
        = R_k\left(-\theta\right)R_k\left(\theta\right)
        = I_2
    \end{equation*}
    \begin{equation*}
        {R_k\left(\theta\right)}^{-1}
        =
        R_k\left(-\theta\right)
        \qedhere
    \end{equation*}
\end{proof}
\begin{definition}\label{M:Position}
    \textit{Position matrix} is defined recursively as
    \begin{align*}
        P_{k,n}\left(\left\{\tensor{\theta}{^1},\dots,\tensor{\theta}{^n}\right\}\right) & \defeq
        \begin{bmatrix}
            P_{k,n-1}\left(\left\{\tensor{\theta}{^1},\dots,\tensor{\theta}{^{n-1}}\right\}\right) & 0_{n\times 1} \\
            0_{1\times n}                                                                              & 1             \\
        \end{bmatrix}
        T_{2,n+1}
        \begin{bmatrix}
            R_k\left(\tensor{\theta}{^n}\right) & 0_{{2}\times{n-1}} \\
            0_{{n-1}\times{2}}                  & {I}_{n-1}          \\
        \end{bmatrix}
        T_{2,n+1} \text{,}                                                                                    \\
        P_{k,0}                                                                          & \defeq I_1\text{,}
    \end{align*}
    where $\theta=\left\{\tensor{\theta}{^i}\right\}\in\R^n$ for $i\in\range{1}{n}$.
\end{definition}
\begin{definition}\label{M:Position:Set}
    Let $P\left(n,k\right)$ be set of position matrices.
\end{definition}
\begin{corollary}[Position matrix at zero]\label{M:Position:Set:Identity}
    \begin{equation*}
        P_{k,n}
        \left(0_{n}\right)
        =
        I_{n+1}
    \end{equation*}
\end{corollary}
\begin{proof}[\proofof{M:Position:Set:Identity}]
    Prove by mathematical induction on $n$,
    Let
    \begin{equation}\label{M:Position:Set:Identity:Proof:Induction}
        P_{k,n-1}\left(0_{n-1}\right)=I_n
    \end{equation}
    \begin{align*}
        P_{k, n}\left(0_{n}\right)
                   & =
        \begin{bmatrix}
            P_{k,n-1}\left(0_{n-1}\right) & 0_{n\times 1} \\
            0_{1\times n}                 & 1             \\
        \end{bmatrix}
        T_{2, n+1}
        \begin{bmatrix}
            R_{k}\left(0\right) & 0_{{2}\times{n-1}} \\
            0_{{n-1}\times{2}}  & {I}_{n-1}          \\
        \end{bmatrix}
        T_{2, n+1} & \text{(\cref{M:Position})}                              \\
                   & =
        \begin{bmatrix}
            I_{n}         & 0_{n\times 1} \\
            0_{1\times n} & 1             \\
        \end{bmatrix}
        T_{2, n+1}
        \begin{bmatrix}
            R_{k}\left(0\right) & 0_{{2}\times{n-1}} \\
            0_{{n-1}\times{2}}  & {I}_{n-1}          \\
        \end{bmatrix}
        T_{2, n+1} & \text{(\cref{M:Position:Set:Identity:Proof:Induction})} \\
                   & =
        \begin{bmatrix}
            I_{n}         & 0_{n\times 1} \\
            0_{1\times n} & 1             \\
        \end{bmatrix}
        T_{2, n+1}
        \begin{bmatrix}
            I_{2}              & 0_{{2}\times{n-1}} \\
            0_{{n-1}\times{2}} & {I}_{n-1}          \\
        \end{bmatrix}
        T_{2, n+1} & \text{(\cref{M:Rotation:Identity})}                     \\
                   & =
        I_{n+1}
        T_{2, n+1}
        I_{n+1}
        T_{2, n+1} & \text{(\cref{Matrix:Identity:Block})}                   \\
                   & =
        T_{2, n+1}
        T_{2, n+1} & \text{(\cref{Matrix:Identity})}                         \\
                   & =
        I_{n+1}    & \text{(\cref{Matrix:Permutation:Square})}
    \end{align*}
    \begin{equation*}
        P_{k,n}
        \left(0_{n}\right)
        =
        I_{n+1}
        \qedhere
    \end{equation*}
\end{proof}
\begin{definition}\label{M:Orientation}
    \textit{Orientation matrix} is defined as
    \begin{align*}
        Q^\pm_n\left(\phi_{n-1},\phi_{n-2},\dots,\phi_1\right) & \defeq
        \begin{bmatrix}
            1             & 0_{1\times n}                                                 \\
            0_{n\times 1} & X^\pm_{+1,n-1}\left(\phi_{n-1},\phi_{n-2},\dots,\phi_1\right) \\
        \end{bmatrix}\text{,}   \\
        Q^\pm_0                                                & \defeq \pm I_1\text{,}
    \end{align*}
    where $\phi_m\in\R^m \text{for } m\in\range{1}{n-1}$.
\end{definition}
\begin{definition}\label{M:Orientation:Set}
    Let $Q\left(n\right)$ be set of orientation matrices.
\end{definition}
\begin{corollary}[Orientation matrix at zero]\label{M:Orientation:Set:Identity}
    \begin{equation*}
        Q^{+}_{n}
        \left(0_{n-1}, 0_{n-2}, \dots\right)
        =
        I_{n+1}
    \end{equation*}
\end{corollary}
\begin{proof}[\proofof{M:Orientation:Set:Identity}]
    Prove by mathematical induction on $n$,
    Let
    \begin{equation}\label{M:Orientation:Set:Identity:Proof:Induction}
        Q^+_{n-1}\left(0_{n-2}, 0_{n-3}, \dots\right)=I_n
    \end{equation}
    \begin{align*}
        Q^+_n\left(0_{n-1}, 0_{n-2}, \dots\right)
                                                                                                       & =
        \begin{bmatrix}
            1             & 0_{1\times n}                                    \\
            0_{n\times 1} & X^\pm_{+1,n-1}\left(0_{n-1},0_{n-2},\dots\right) \\
        \end{bmatrix}                            & \text{(\cref{M:Orientation})}                                                               \\
                                                                                                       & =
        \begin{bmatrix}
            1             & 0_{1\times n}                                                               \\
            0_{n\times 1} & P_{+1,n-1}\left(0_{n-1}\right)Q^+_{n-1}\left(0_{n-2}, 0_{n-3}, \dots\right) \\
        \end{bmatrix} & \text{(\cref{M:Point})}                                            \\
                                                                                                       & =
        \begin{bmatrix}
            1             & 0_{1\times n}                      \\
            0_{n\times 1} & P_{+1,n-1}\left(0_{n-1}\right) I_n \\
        \end{bmatrix}                                          & \text{(\cref{M:Orientation:Set:Identity:Proof:Induction})}                    \\
                                                                                                       & =
        \begin{bmatrix}
            1             & 0_{1\times n}                  \\
            0_{n\times 1} & P_{+1,n-1}\left(0_{n-1}\right) \\
        \end{bmatrix}                                              & \text{(\cref{Matrix:Identity})}                                           \\
                                                                                                       & =
        \begin{bmatrix}
            1             & 0_{1\times n} \\
            0_{n\times 1} & I_{n}         \\
        \end{bmatrix}                                                               & \text{(\cref{M:Position:Set:Identity})}                  \\
                                                                                                       & =
        I_{n+1}                                                                                        & \text{(\cref{Matrix:Identity:Block})}
    \end{align*}
    \begin{equation*}
        Q^+_n
        \left(0_{n-1}, 0_{n-2}, \dots\right)
        =
        I_{n+1}
        \qedhere
    \end{equation*}
\end{proof}
\begin{definition}\label{M:Point}
    \textit{Point matrix} is defined as
    \begin{align*}
        X^\pm_{k,n}\left(\theta,\phi_{n-1},\phi_{n-2},\dots,\phi_1\right) & \defeq
        P_{k,n}\left(\theta\right)
        Q^\pm_n\left(\phi_{n-1},\phi_{n-2},\dots,\phi_1\right)\text{,}
    \end{align*}
    where $\theta\in\R^n$ and $\phi_m\in\R^m \text{for } m\in\range{1}{n-1}$.
\end{definition}
\begin{definition}\label{M:Point:Set}
    Let $X\left(n,k\right)$ be set of point matrices.
\end{definition}
\begin{corollary}[Position matrix as subset of point matrix]\label{M:Point:Position}
    \begin{equation*}
        P\left(k,n\right)\subset X\left(k,n\right)
    \end{equation*}
\end{corollary}
\begin{proof}[\proofof{M:Point:Position}]
    \begin{align*}
        \forall{P \in P\left(k,n\right)}
        \forall{Q\in Q\left(n\right)},
                                  &
        P Q \in X\left(k,n\right) & \text{(\cref{M:Point})}                    \\
        \implies                  &
        P I \in X\left(k,n\right) & \text{(\cref{M:Orientation:Set:Identity})} \\
        \implies                  &
        P \in X\left(k,n\right)   & \text{(\cref{Matrix:Identity})}
    \end{align*}
    \begin{equation*}
        P\left(k,n\right)\subset X\left(k,n\right) \qedhere
    \end{equation*}
\end{proof}
\begin{corollary}[Point matrix at zero]\label{M:Point:Set:Identity}
    \begin{equation*}
        X^{+}_{k,n}
        \left(0_{n}, 0_{n-1}, \dots\right)
        =
        I_{n+1}
    \end{equation*}
\end{corollary}
\begin{proof}[\proofof{M:Point:Set:Identity}]
    It can be implied from \cref{M:Position:Set:Identity,M:Orientation:Set:Identity}.
\end{proof}
\subsection{Group structure}
\begin{proposition}
    Group of position matrix with multiplication is a subgroup of an orthogonal group for $k>0$.
    \begin{equation*}
        \left(P\left(n, k\right),\cdot\right)\cong O\left(n+1\right)
    \end{equation*}
\end{proposition}
\begin{proof}
    \begin{align*}
        P&=P_{k,n}\left(\left\{\tensor{\theta}{^1},\dots,\tensor{\theta}{^n}\right\}\right) \\
        &\defeq
        \begin{bmatrix}
            P_{k,n-1}\left(\left\{\tensor{\theta}{^1_{n}},\dots,\tensor{\theta}{^{n-1}}\right\}\right) & 0_{n\times 1} \\
            0_{1\times n}                                                                              & 1             \\
        \end{bmatrix}
        T_{2,n+1}
        \begin{bmatrix}
            R_k\left(\tensor{\theta}{^n}\right) & 0_{{2}\times{n-1}} \\
            0_{{n-1}\times{2}}                  & {I}_{n-1}          \\
        \end{bmatrix}
        T_{2,n+1} && \text{(\cref{M:Position})} \\
        P^T
        &=
        T_{2,n+1}^T
        \begin{bmatrix}
            R_k\left(\tensor{\theta}{^n}\right) & 0_{{2}\times{n-1}} \\
            0_{{n-1}\times{2}}                  & {I}_{n-1}          \\
        \end{bmatrix}^T
        T_{2,n+1}^T
        \begin{bmatrix}
            P_{k,n-1}\left(\left\{\tensor{\theta}{^1_{n}},\dots,\tensor{\theta}{^{n-1}}\right\}\right) & 0_{n\times 1} \\
            0_{1\times n}                                                                              & 1             \\
        \end{bmatrix}^T && \text{(Transpose of product)} \\
        &=
        T_{2,n+1}^T
        \begin{bmatrix}
            R_k\left(\tensor{\theta}{^n}\right)^T & 0_{{2}\times{n-1}} \\
            0_{{n-1}\times{2}}                  & {I}_{n-1}          \\
        \end{bmatrix}
        T_{2,n+1}^T
        \begin{bmatrix}
            P_{k,n-1}\left(\left\{\tensor{\theta}{^1_{n}},\dots,\tensor{\theta}{^{n-1}}\right\}\right)^T & 0_{n\times 1} \\
            0_{1\times n}                                                                              & 1             \\
        \end{bmatrix} && \text{(Transpose of block)} \\
        &=
        T_{2,n+1}
        \begin{bmatrix}
            R_k\left(\tensor{\theta}{^n}\right)^T & 0_{{2}\times{n-1}} \\
            0_{{n-1}\times{2}}                  & {I}_{n-1}          \\
        \end{bmatrix}
        T_{2,n+1}
        \begin{bmatrix}
            P_{k,n-1}\left(\left\{\tensor{\theta}{^1_{n}},\dots,\tensor{\theta}{^{n-1}}\right\}\right)^T & 0_{n\times 1} \\
            0_{1\times n}                                                                              & 1             \\
        \end{bmatrix} && \text{(Transpose of permutation matrix)} \\
        PP^T
        &=
        \left(\begin{bmatrix}
            P_{k,n-1}\left(\left\{\tensor{\theta}{^1_{n}},\dots,\tensor{\theta}{^{n-1}}\right\}\right) & 0_{n\times 1} \\
            0_{1\times n}                                                                              & 1             \\
        \end{bmatrix}
        T_{2,n+1}
        \begin{bmatrix}
            R_k\left(\tensor{\theta}{^n}\right) & 0_{{2}\times{n-1}} \\
            0_{{n-1}\times{2}}                  & {I}_{n-1}          \\
        \end{bmatrix}
        T_{2,n+1}\right)\\
        &\left(T_{2,n+1}
        \begin{bmatrix}
            R_k\left(\tensor{\theta}{^n}\right)^T & 0_{{2}\times{n-1}} \\
            0_{{n-1}\times{2}}                  & {I}_{n-1}          \\
        \end{bmatrix}
        T_{2,n+1}
        \begin{bmatrix}
            P_{k,n-1}\left(\left\{\tensor{\theta}{^1_{n}},\dots,\tensor{\theta}{^{n-1}}\right\}\right)^T & 0_{n\times 1} \\
            0_{1\times n}                                                                              & 1             \\
        \end{bmatrix}\right) && \text{(Equation above)} \\
        &=
        \begin{bmatrix}
            P_{k,n-1}\left(\left\{\tensor{\theta}{^1_{n}},\dots,\tensor{\theta}{^{n-1}}\right\}\right) & 0_{n\times 1} \\
            0_{1\times n}                                                                              & 1             \\
        \end{bmatrix}
        T_{2,n+1}
        \begin{bmatrix}
            R_k\left(\tensor{\theta}{^n}\right) & 0_{{2}\times{n-1}} \\
            0_{{n-1}\times{2}}                  & {I}_{n-1}          \\
        \end{bmatrix}\\
        &
        \begin{bmatrix}
            R_k\left(\tensor{\theta}{^n}\right)^T & 0_{{2}\times{n-1}} \\
            0_{{n-1}\times{2}}                  & {I}_{n-1}          \\
        \end{bmatrix}
        T_{2,n+1}
        \begin{bmatrix}
            P_{k,n-1}\left(\left\{\tensor{\theta}{^1_{n}},\dots,\tensor{\theta}{^{n-1}}\right\}\right)^T & 0_{n\times 1} \\
            0_{1\times n}                                                                              & 1             \\
        \end{bmatrix} && \text{()} \\
        &=
        \begin{bmatrix}
            P_{k,n-1}\left(\left\{\tensor{\theta}{^1_{n}},\dots,\tensor{\theta}{^{n-1}}\right\}\right) & 0_{n\times 1} \\
            0_{1\times n}                                                                              & 1             \\
        \end{bmatrix}
        T_{2,n+1}
        \begin{bmatrix}
            R_k\left(\tensor{\theta}{^n}\right)R_k\left(\tensor{\theta}{^n}\right)^T & 0_{{2}\times{n-1}} \\
            0_{{n-1}\times{2}}                  & {I}_{n-1}          \\
        \end{bmatrix}\\
        &
        T_{2,n+1}
        \begin{bmatrix}
            P_{k,n-1}\left(\left\{\tensor{\theta}{^1_{n}},\dots,\tensor{\theta}{^{n-1}}\right\}\right)^T & 0_{n\times 1} \\
            0_{1\times n}                                                                              & 1             \\
        \end{bmatrix} && \text{(\cref{Matrix:Product:Block})} \\
        &=
        \begin{bmatrix}
            P_{k,n-1}\left(\left\{\tensor{\theta}{^1_{n}},\dots,\tensor{\theta}{^{n-1}}\right\}\right) & 0_{n\times 1} \\
            0_{1\times n}                                                                              & 1             \\
        \end{bmatrix}
        T_{2,n+1}\\
        &
        T_{2,n+1}
        \begin{bmatrix}
            P_{k,n-1}\left(\left\{\tensor{\theta}{^1_{n}},\dots,\tensor{\theta}{^{n-1}}\right\}\right)^T & 0_{n\times 1} \\
            0_{1\times n}                                                                              & 1             \\
        \end{bmatrix} && \text{()} \\
        &=
        \begin{bmatrix}
            P_{k,n-1}\left(\left\{\tensor{\theta}{^1_{n}},\dots,\tensor{\theta}{^{n-1}}\right\}\right) & 0_{n\times 1} \\
            0_{1\times n}                                                                              & 1             \\
        \end{bmatrix}
        \begin{bmatrix}
            P_{k,n-1}\left(\left\{\tensor{\theta}{^1_{n}},\dots,\tensor{\theta}{^{n-1}}\right\}\right)^T & 0_{n\times 1} \\
            0_{1\times n}                                                                              & 1             \\
        \end{bmatrix} && \text{()} \\
        &=
        \begin{bmatrix}
            P_{k,n-1}\left(\left\{\tensor{\theta}{^1_{n}},\dots,\tensor{\theta}{^{n-1}}\right\}\right)P_{k,n-1}\left(\left\{\tensor{\theta}{^1_{n}},\dots,\tensor{\theta}{^{n-1}}\right\}\right)^T & 0_{n\times 1} \\
            0_{1\times n}                                                                              & 1             \\
        \end{bmatrix} && \text{(\cref{Matrix:Product:Block})} \\
        &= I_{n+1} && \text{(mathematical induction)}
    \end{align*}
\end{proof}
\begin{proposition}
    Group of position matrix with multiplication is a subgroup of an $\left(1,n\right)$-orthochronus indefinite orthogonal group for $k<0$.
    \begin{equation*}
        \left(P\left(n, k\right),\cdot\right)\cong O^+\left(1,n\right)
    \end{equation*}
\end{proposition}
\begin{proof}
    \begin{align*}
        P&=P_{k,n}\left(\left\{\tensor{\theta}{^1},\dots,\tensor{\theta}{^n}\right\}\right) \\
        &\defeq
        \begin{bmatrix}
            P_{k,n-1}\left(\left\{\tensor{\theta}{^1_{n}},\dots,\tensor{\theta}{^{n-1}}\right\}\right) & 0_{n\times 1} \\
            0_{1\times n}                                                                              & 1             \\
        \end{bmatrix}
        T_{2,n+1}
        \begin{bmatrix}
            R_k\left(\tensor{\theta}{^n}\right) & 0_{{2}\times{n-1}} \\
            0_{{n-1}\times{2}}                  & {I}_{n-1}          \\
        \end{bmatrix}
        T_{2,n+1} && \text{(\cref{M:Position})} \\
        P^T
        &=
        T_{2,n+1}^T
        \begin{bmatrix}
            R_k\left(\tensor{\theta}{^n}\right) & 0_{{2}\times{n-1}} \\
            0_{{n-1}\times{2}}                  & {I}_{n-1}          \\
        \end{bmatrix}^T
        T_{2,n+1}^T
        \begin{bmatrix}
            P_{k,n-1}\left(\left\{\tensor{\theta}{^1_{n}},\dots,\tensor{\theta}{^{n-1}}\right\}\right) & 0_{n\times 1} \\
            0_{1\times n}                                                                              & 1             \\
        \end{bmatrix}^T && \text{(Transpose of product)} \\
        &=
        T_{2,n+1}^T
        \begin{bmatrix}
            R_k\left(\tensor{\theta}{^n}\right)^T & 0_{{2}\times{n-1}} \\
            0_{{n-1}\times{2}}                  & {I}_{n-1}          \\
        \end{bmatrix}
        T_{2,n+1}^T
        \begin{bmatrix}
            P_{k,n-1}\left(\left\{\tensor{\theta}{^1_{n}},\dots,\tensor{\theta}{^{n-1}}\right\}\right)^T & 0_{n\times 1} \\
            0_{1\times n}                                                                              & 1             \\
        \end{bmatrix} && \text{(Transpose of block)} \\
        &=
        T_{2,n+1}
        \begin{bmatrix}
            R_k\left(\tensor{\theta}{^n}\right)^T & 0_{{2}\times{n-1}} \\
            0_{{n-1}\times{2}}                  & {I}_{n-1}          \\
        \end{bmatrix}
        T_{2,n+1}
        \begin{bmatrix}
            P_{k,n-1}\left(\left\{\tensor{\theta}{^1_{n}},\dots,\tensor{\theta}{^{n-1}}\right\}\right)^T & 0_{n\times 1} \\
            0_{1\times n}                                                                              & 1             \\
        \end{bmatrix} && \text{(Transpose of permutation matrix)} \\
        g
        &= \diag{-1,1,\dots,1} \\
        gPgP^T
        &= I_{n+1} && \text{(mathematical induction)}
    \end{align*}
    \begin{align*}
        \tensor{P}{^1_1} &> 0
    \end{align*}
\end{proof}
\begin{proposition}
    Group of position matrix with multiplication is isomorphic to translation group for $k\to0$.
    \begin{equation*}
        \left(P\left(n, k\right),\cdot\right)\cong E\left(n\right)
    \end{equation*}
\end{proposition}
\begin{proof}
    \begin{align*}
        R_k\left(\theta\right)
        &\defeq\begin{bmatrix}
            \cos_k\left(\theta\right) & -\sin_k^\ast\left(\theta\right) \\
            \sin_k\left(\theta\right) & \cos_k\left(\theta\right) \\
        \end{bmatrix} && \text{(\cref{M:Rotation})} \\
        &\to\begin{bmatrix}
            1 & 0 \\
            k\theta & 1 \\
        \end{bmatrix} && \text{(Limits of the functions)} \\
        P_{k,n}\left(\left\{\tensor{\theta}{^1},\dots,\tensor{\theta}{^n}\right\}\right)
        &\defeq
        \begin{bmatrix}
            P_{k,n-1}\left(\left\{\tensor{\theta}{^1_{n}},\dots,\tensor{\theta}{^{n-1}}\right\}\right) & 0_{n\times 1} \\
            0_{1\times n}                                                                              & 1             \\
        \end{bmatrix}
        T_{2,n+1}
        \begin{bmatrix}
            R_k\left(\tensor{\theta}{^n}\right) & 0_{{2}\times{n-1}} \\
            0_{{n-1}\times{2}}                  & {I}_{n-1}          \\
        \end{bmatrix}
        T_{2,n+1} && \text{(\cref{M:Position})} \\
        &\to
        \begin{bmatrix}
            P_{k,n-1}\left(\left\{\tensor{\theta}{^1_{n}},\dots,\tensor{\theta}{^{n-1}}\right\}\right) & 0_{n\times 1} \\
            0_{1\times n}                                                                              & 1             \\
        \end{bmatrix}
        T_{2,n+1}
        \begin{bmatrix}
            \begin{matrix*}1&0\\k\tensor{\theta}{^n}&0\\\end{matrix*} & 0_{{2}\times{n-1}} \\
            0_{{n-1}\times{2}}                  & {I}_{n-1}          \\
        \end{bmatrix}
        T_{2,n+1} && \text{(Equation above)} \\
        &=
        \begin{bmatrix}
            P_{k,n-1}\left(\left\{\tensor{\theta}{^1_{n}},\dots,\tensor{\theta}{^{n-1}}\right\}\right) & 0_{n\times 1} \\
            0_{1\times n}                                                                              & 1             \\
        \end{bmatrix}
        \begin{bmatrix}
            {I}_{n} & 0_{n \times1}          \\
            \begin{matrix*}k\tensor{\theta}{^n} & 0_{1\times n-1}\end{matrix*} & 1 \\
        \end{bmatrix} && \text{(property of permutation matrix)} \\
        &=
        \begin{bmatrix}
            P_{k,n-1}\left(\left\{\tensor{\theta}{^1_{n}},\dots,\tensor{\theta}{^{n-1}}\right\}\right) & 0_{n \times1}          \\
            \begin{matrix*}k\tensor{\theta}{^n} & 0_{1\times n-1}\end{matrix*} & 1 \\
        \end{bmatrix} && \text{(\cref{Matrix:Product:Block})} \\
        P_{k,n}\left(\tensor{\theta}{}\right)
        &\to
        \begin{bmatrix}
            1 & 0_{n \times1}          \\
            k\tensor{\theta}{} & I_n \\
        \end{bmatrix} && \text{(mathematical induction)} \\
        P_{k,n}\left(\tensor{\theta}{}\right) P_{k,n}\left(\tensor{\phi}{}\right)
        &\to
        \begin{bmatrix}
            1 & 0_{n \times1}          \\
            k\tensor{\theta}{} & I_n \\
        \end{bmatrix}\begin{bmatrix}
            1 & 0_{n \times1}          \\
            k\tensor{\phi}{} & I_n \\
        \end{bmatrix} && \text{(Equation above)} \\
        &=
        \begin{bmatrix}
            1 & 0_{n \times1}          \\
            k\tensor{\theta}{}+k\tensor{\phi}{} & I_n \\
        \end{bmatrix} && \text{(\cref{Matrix:Product:Block})}
    \end{align*}
\end{proof}
\begin{proposition}
    Group of orienatation with multiplication is isomorphic to orthogonal group.
    \begin{equation*}
        \left(Q\left(n\right),\cdot\right)\cong O\left(n\right)
    \end{equation*}
\end{proposition}
\begin{proposition}
    Group of point with multiplication is isomorphic to orthogonal group for $k>0$.
    \begin{equation*}
        \left(X\left(n\right),\cdot\right)\cong O\left(n\right)
    \end{equation*}
\end{proposition}
\begin{proposition}
    Group of point with multiplication is isomorphic to $\left(1,n\right)$-orthochronus indefinite orthogonal group for $k<0$.
    \begin{equation*}
        \left(X\left(n\right),\cdot\right)\cong O^+\left(1, n\right)
    \end{equation*}
\end{proposition}
\begin{proposition}
    Group of point with multiplication is isomorphic to Euclidean group for $k\to0$.
    \begin{equation*}
        \left(X\left(n\right),\cdot\right)\cong E\left(n\right)
    \end{equation*}
\end{proposition}

\end{document}