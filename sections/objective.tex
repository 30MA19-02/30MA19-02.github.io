% !TeX root = ../main.tex

\documentclass[../main.tex]{subfiles}

\begin{document}
\section{Objective}
\paragraph{Objective}\label{Objective}
The objective is that
given a natural number $n$ and a real number $\kappa$,
one can construct
\begin{APAenumerate}
    \item a set $M$
    \item a Lie group $M$ with operation $\otimes_M$,
    \item an $n$-dimensional $C^\infty$-manifold $M$ with chart $\varphi_i \subset M\to\R^n$,
    \item a Riemannian manifold $M$ with inner product $g_p\in T_pM\times T_pM\to\R$
\end{APAenumerate}
(to be determined)
such that
\begin{APAitemize}
    \item group action is distance preserved.
    \item model is continuous with respect to $\kappa$ (and smooth with respect to each basis).
    \item for all two-dimensional linear subspaces of the manifold, the sectional curvature is $\kappa$.
\end{APAitemize}
(cannot figure formal definition out yet.)

\begin{conjecture}\label{GeometricGroupStructure}
    If the parameters $\left(\kappa,n\right)$ is associated with Klein geometry $\left(G,H\right)$
    then $\left(M,\otimes_M\right)\cong G/H$.

    That is, from \cref{KleinGeometryExample},
    \begin{APAitemize}
        \item For $\kappa>0$, $G\cong O\left(n+1\right)$ and $H\cong O\left(n\right)$.
        \item For $\kappa=0$, $G\cong E\left(n\right)$ and $H\cong O\left(n\right)$.
        \item For $\kappa<0$, $G\cong O^{+}\left(1,n\right)$ and $H\cong O\left(n\right)$.
    \end{APAitemize}
\end{conjecture}
\end{document}