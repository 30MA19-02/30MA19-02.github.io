% !TeX root = ../main.tex
\documentclass[../main.tex]{subfiles}
\begin{document}
\section{Literature Review}
\begin{definition}[Group {\autocite{mathworld}}]\label{Group}
    A group \(G\) is a finite or infinite set of elements
    together with a binary operation (called the group operation)
    that together satisfy the four fundamental properties of
    closure, associativity, the identity property, and the inverse property.
    The operation with respect to which a group is defined is often called the "group operation,"
    and a set is said to be a group "under" this operation.
    Elements \(A,B,C,\dots\)
    with binary operation between \(A\) and \(B\) denoted \(AB\) form a group if
    \begin{APAenumerate}
        \item Closure: If \(A\) and \(B\) are two elements in \(G\), then the product \(AB\) is also in \(G\).
        \item Associativity: The defined multiplication is associative, i.e., for all \(A,B,C \in G\), \(\left(AB\right)C=A\left(BC\right)\).
        \item Identity: There is an identity element \(I\) (a.k.a. \(1\), \(E\), or \(e\)) such that \(IA=AI=A\) for every element \(A \in G\).
        \item Inverse: There must be an inverse (a.k.a. reciprocal) of each element.
        Therefore, for each element \(A\) of \(G\), the set contains an element \(B=A^{-1}\) such that \(AA^{-1}=A^{-1}A=I\).
    \end{APAenumerate}
\end{definition}
\begin{definition}[Lie group {\autocite[][Chapter~7]{lee_2013}}]\label{LieGroup}
    A \textit{Lie group} is a smooth manifold \(G\) (without boundary)
    that is also a group in the algebric sense,
    with the property that
    the multiplication map \(m\colon G\times G\to G\)
    and inversion map \(i\colon G\to G\),
    given by
    \begin{equation*}
        m\left(g,h\right) = gh\text{,}
        i\left(g\right) = g^{-1}\text{.}
    \end{equation*}
    are both smooth.
\end{definition}
\begin{definition}[Group action {\autocite[][Chapter~7]{lee_2013}}]\label{GroupAction}
    If \(G\) is a group and \(M\) is a set,
    a \textit{left action of \(G\) on \(M\)}
    is a map \(G \times M \to M\),
    often written as \(\left(g,p\right)\mapsto g \cdot p\),
    that satisfies
    \begin{APAitemize}
        \item \(g_1 \cdot \left(g_2 \cdot p\right)=\left(g_1 g_2\right) \cdot p\) for all \(g_1,g_2 \in G\) and \(p \in M\);
        \item \(e \cdot p = p\) for all \(p \in M\).
    \end{APAitemize}
    A \textit{right action} is defined analogously
    as a map \(M \times G \to M\)
    with the appropriate composition law:
    \begin{APAitemize}
        \item \(\left(p \cdot g_1\right) \cdot g_2 = p \cdot \left(g_1 g_2\right)\) for all \(g_1,g_2 \in G\) and \(p \in M\);
        \item \(p \cdot e = p\) for all \(p \in M\).
    \end{APAitemize}
\end{definition}
\begin{definition}[Abstract differentiable manifold {\autocite[][Chapter~5A]{kuhnelwolfgang_2006}}]\label{Manifold}
    A \textit{\(k\)-dimensional differentiable manifold} (briefly: a \(k\)-manifold)
    is a set \(M\) together with a family \(\left(M_i\right)_{i\in I}\) of subsets such that
    \begin{APAenumerate}
        \item \(M=\bigcup_{i\in I} M_i\) (union),
        \item for every \(i\in I\) there is an injective map \(\varphi_i\colon M_i\to\mathbb{R}^k\) so that \(\phi_i\left(M_i\right)\) is open in \(\mathbb{R}^k\), and
        \item for \(M_i\cap M_j\ne\emptyset\), \(\varphi_i\left(M_i\cap M_j\right)\) is open in \(\mathbb{R}^k\).
    \end{APAenumerate}
\end{definition}
\begin{definition}[Structures on a manifold {\autocite[][Chapter~5A]{kuhnelwolfgang_2006}}]\label{SmoothManifold}
    Given a \(k\)-dimensional differentiable manifold,
    one gets additional structure
    by replacing aditional requirements on the transformation functions \(\varphi_j\circ\varphi_i^{-1}\),
    which belong to the atlas of the manifold;
    if all \(\varphi_j\circ\varphi_i^{-1}\) are (left-hand side),
    then one speaks of (right-hand side) as follows:
    \begin{center}
        \begin{tabular}{ r c l }
            continuous                  & \(\iff\) & topological manifold                                   \\
            differentiable              & \(\iff\) & differentiable manifold                                \\
            \(C^1\)-differentiable      & \(\iff\) & \(C^1\)-manifold                                       \\
            \(C^r\)-differentiable      & \(\iff\) & \(C^r\)-manifold                                       \\
            \(C^\infty\)-differentiable & \(\iff\) & \(C^\infty\)-manifold                                  \\
            real analytic               & \(\iff\) & real analytic manifold                                 \\
            complex analytic            & \(\iff\) & complex analytic manifold of dimension \(\frac{k}{2}\) \\
            affine                      & \(\iff\) & affine manifold                                        \\
            projective                  & \(\iff\) & projective manifold                                    \\
            conformal                   & \(\iff\) & manifold with a conformal structure                    \\
            orienatation-preserving     & \(\iff\) & orientable manifold                                    \\
        \end{tabular}
    \end{center}
\end{definition}
\begin{definition}[Riemannian metric {\autocite[][Chapter~5C]{kuhnelwolfgang_2006}}]\label{RiemannianMetric}
    A \textit{Riemannian metric} \(g\) on \(M\)
    is an association \(p\mapsto g_p\in L^2\left(T_pM;\mathbb{R}\right)\)
    such that the following conditins are satisfied:
    \begin{APAenumerate}
        \item \(g_p\left(X,Y\right)=g_p\left(Y,X\right)\) for all \(X\), \(Y\), \hfill (\textit{symmetry})
        \item \(g_p\left(X,X\right)>0\) for all \(X\ne0\), \hfill (\textit{positive definiteness})
        \item The coefficient \(\tensor{g}{_i_j}\) in every local representation (i.e., in every chart) \[g_p=\sum_{i,j}\tensor{g}{_i_j}\left(p\right)\cdot \left.d\tensor{x}{^i}\right|_p \otimes\left.d\tensor{x}{^j}\right|_p\] are differentiable functions. \hfill (\textit{differentiability})
    \end{APAenumerate}
\end{definition}
\begin{remark}\label{RiemannianMannifold}
    The pair \(\left(M,g\right)\) is then called \textit{Riemannian manifold}.
    One also refers to the Riemannian metric as the \textit{metric tensor}.
    In local coordinates the metric tensor is given by the matrix (\(\tensor{g}{_i_j}\)) of functions.
    In Ricci calculus this is simply written as \(\tensor{g}{_i_j}\)
\end{remark}
\begin{remark}\label{Norm_and_Angle_as_RiemannianMetric}
    A Riemannian metric \(g\) defines at every point \(p\)
    an \textit{inner product} \(g_p\) on the tangent space \(T_pM\),
    and therefore the notation \(\left\langle X,Y\right\rangle\) instead of \(g_p\left(X,Y\right)\) is also used.
    The notions of angles and lengths are determined by this inner product,
    just as these notions are determined by the first fundamental form on surface elements.
    The length or norm of vector \(X\) is given by \(\norm{ X}\coloneqq\sqrt{g\left(X,X\right)}\),
    and the angle \(\beta\) between two tangent vectors \(X\) and \(Y\)
    can be defined by the validity of the equation \(\cos\beta\cdot\norm{X}\cdot\norm{Y}=g\left(X,Y\right)\).
\end{remark}
\begin{definition}[Christoffel symbols of the Second Kind {\autocite{mathworld}}]\label{ChristoffelSymbol}
    \begin{align*}
        \tensor{\Gamma}{^m_i_j}
         & \coloneqq \tensor{\epsilon}{^m}
        \cdot\frac{\partial \tensor{\epsilon}{_i}}{\partial \tensor{q}{^j}} \\
         & = \tensor{g}{^k^m}\tensor{\Gamma}{_k_i_j}                        \\
         & = \frac{1}{2}\tensor{g}{^k^m}\left(
        \frac{\partial \tensor{g}{_i_k}}{\partial \tensor{q}{^j}}
        + \frac{\partial \tensor{g}{_j_k}}{\partial \tensor{q}{^i}}
        - \frac{\partial \tensor{g}{_i_j}}{\partial \tensor{q}{^k}}
        \right)
    \end{align*}
    where \(\frac{\partial}{\partial x}\) is a partial derivative,
    \(\tensor{g}{^k^m}\) is the metric tensor,
    \begin{equation*}
        \tensor{\epsilon}{_i} = \frac{\partial \mathbf{r}}{\partial \tensor{q}{^i}}
    \end{equation*}
    where \(\mathbf{r}\) is the radius vector, and
    \begin{equation*}
        \tensor{\epsilon}{^i} = \tensor{g}{^i^j}\tensor{\epsilon}{_j} \text{.}
    \end{equation*}
\end{definition}
\begin{definition}[Curvature tensor {\autocite[][Chapter~4C]{kuhnelwolfgang_2006}}]\label{CurvatureTensor}
    \[
        R\left(X,Y\right)Z
        \coloneqq
        \nabla_X\nabla_YZ - \nabla_Y\nabla_XZ - \nabla_{\left[X,Y\right]}Z
    \]
    is a tensor field, which is called the \textit{curvature tensor} of the surface.
\end{definition}
\begin{remark}\label{CurvatureTensor_as_ChristoffelSymbol}
    The left-hand side of the Gauss equation is called the \textit{curvature tensor}
    and is in general expressed in the form
    \[
        \tensor{R}{^s_i_k_j}
        \coloneqq
        \frac{\partial}{\partial\tensor{u}{^k}}\tensor{\Gamma}{^s_i_j}
        - \frac{\partial}{\partial\tensor{u}{^j}}\tensor{\Gamma}{^s_i_k}
        + \sum_{r}\left(\tensor{\Gamma}{^r_i_j}\tensor{\Gamma}{^s_r_k}-\tensor{\Gamma}{^r_i_k}\tensor{\Gamma}{^s_r_j}\right) \text{.}
    \]
\end{remark}
\begin{definition}[Sectional Curvature {\autocite[][Chapter~6B]{kuhnelwolfgang_2006}}]\label{SectionalCurvature}
    With respect to a given Riemannian metric \(\left\langle\cdot,\cdot\right\rangle\),
    the \textit{standard curvature tensor} \(\tensor{{R_1}}{}\) is defined by the relation
    \(R_1\left(X,Y\right)Z\coloneqq\left\langle Y,Z\right\rangle X-\left\langle X,Z\right\rangle Y\).
    We then set
    \begin{align*}
        \kappa_1\left(X,Y\right)
        \coloneqq\left\langle R_1\left(X,Y\right)Y,X\right\rangle
        =\left\langle X,X\right\rangle\cdot\left\langle Y,Y\right\rangle-\left\langle X,Y\right\rangle^2\text{,} \\
        \kappa\left(X,Y\right)
        \coloneqq\left\langle R\left(X,Y\right)Y,X\right\rangle\text{.}
    \end{align*}
    Let \(\sigma\subset T_pM\) be a two-dimensional subspace, spanned by \(X\), \(Y\).
    Then the quantity
    \[
        K_\sigma
        \coloneqq
        \frac{\kappa\left(X,Y\right)}{\kappa_1\left(X,Y\right)}
    \]
    is called the \textit{sectional curvature} of the Riemannian manifold with respect to the plane \(\sigma\).
\end{definition}
\end{document}