% !TeX root = ../../main.tex
\documentclass[../methodology.tex]{subfiles}
\begin{document}
\subsection{Model Foundation}
\subsubsection{Trigonometry}
\begin{definition}\label{M:Trigonometry}
    \textit{Generalized trigonometric functions} \(f_\lambda\colon \mathbb{R}\to\mathbb{R}\) and \(f_\lambda^\ast:\mathbb{R}\to\mathbb{R}\) are defined as
    \begin{align*}
        f_\lambda\left(\theta\right)      & \coloneqq
        \begin{cases}
            g\left(\lambda\theta\right) & \text{if \(\lambda\geq0\),} \\
            h\left(\lambda\theta\right) & \text{otherwise,}           \\
        \end{cases} \\
        f_\lambda^\ast\left(\theta\right) & \coloneqq
        \begin{cases}
            g\left(\lambda\theta\right)  & \text{if \(\lambda\geq0\),} \\
            h\left(-\lambda\theta\right) & \text{otherwise,}           \\
        \end{cases}
    \end{align*}
    where \(g\) (resp. \(h\)) are the associated trigonometric (resp. hyperbolic) function.
    (see \cref{TrigonometryPlotted})
\end{definition}
\begin{example}[Generalized sine functions]\label{M:Trigonometry:Sine}
    \begin{align*}
        \sin_\lambda{\theta}
         & \coloneqq
        \begin{cases}
            \sin\left(\lambda\theta\right)  & \text{if \(\lambda\geq0\),} \\
            \sinh\left(\lambda\theta\right) & \text{otherwise.}           \\
        \end{cases}           \\
        \sin_\lambda^\ast{\theta}
         & \coloneqq
        \begin{cases}
            \sin\left(\lambda\theta\right)   & \text{if \(\lambda\geq0\),} \\
            \sinh\left(-\lambda\theta\right) & \text{otherwise.}           \\
        \end{cases}          \\
         & =
        \begin{cases}
            \sin\left(\lambda\theta\right)   & \text{if \(\lambda\geq0\),} \\
            -\sinh\left(\lambda\theta\right) & \text{otherwise.}           \\
        \end{cases}          \\
         & =\begin{cases}
                \sin\left(\abs{\lambda}\theta\right)  & \text{if \(\lambda\geq0\),} \\
                \sinh\left(\abs{\lambda}\theta\right) & \text{otherwise.}           \\
            \end{cases}
    \end{align*}
    (see \cref{TrigonometrySinePlotted,TrigonometrySineVarPlotted})
\end{example}
\begin{example}[Generalized cosine functions]\label{M:Trigonometry:Cosine}
    \begin{align*}
        \cos_\lambda{\theta}
         & \coloneqq
        \begin{cases}
            \cos\left(\lambda\theta\right)  & \text{if \(\lambda\geq0\),} \\
            \cosh\left(\lambda\theta\right) & \text{otherwise.}           \\
        \end{cases}     \\
        \cos_\lambda^\ast{\theta}
         & \coloneqq
        \begin{cases}
            \cos\left(\lambda\theta\right)   & \text{if \(\lambda\geq0\),} \\
            \cosh\left(-\lambda\theta\right) & \text{otherwise.}           \\
        \end{cases}    \\
         & =\begin{cases}
                \cos\left(\lambda\theta\right)  & \text{if \(\lambda\geq0\),} \\
                \cosh\left(\lambda\theta\right) & \text{otherwise.}           \\
            \end{cases} \\
         & =\cos_\lambda{\theta}
    \end{align*}
    (see \cref{TrigonometryCosinePlotted})
\end{example}
\begin{theorem}[Pythagorean's identity equivalence]\label{M:Trigonometry:Pythagorean}
    \begin{align*}
        \cos_\lambda^2{\theta}+\sign{\lambda}\sin_\lambda^2{\theta} & =1 \\
        \sec_\lambda^2{\theta}-\sign{\lambda}\tan_\lambda^2{\theta} & =1 \\
    \end{align*}
\end{theorem}
\begin{proof}[Proof of \cref{M:Trigonometry:Pythagorean}]
    Proof by exhaustion.
\end{proof}
\begin{proposition}[Generalized trigonometric functions of sum of arguments]\label{M:Trigonometry:Sum}
    \begin{align*}
        \sin_\lambda\left(\theta+\phi\right)
         & =\sin_\lambda{\theta}\cos_\lambda{\phi}+\cos_\lambda{\theta}\sin_\lambda{\phi}               \\
        \sin_\lambda^\ast\left(\theta+\phi\right)
         & =\sin_\lambda^\ast{\theta}\cos_\lambda{\phi}+\cos_\lambda{\theta}\sin_\lambda^\ast{\phi}     \\
        \cos_\lambda\left(\theta+\phi\right)
         & =\cos_\lambda{\theta}\cos_\lambda{\phi}-\sign{\lambda}\sin_\lambda{\theta}\sin_\lambda{\phi} \\
         & =\cos_\lambda{\theta}\cos_\lambda{\phi}-\sin_\lambda^\ast{\theta}\sin_\lambda{\phi}          \\
         & =\cos_\lambda{\theta}\cos_\lambda{\phi}-\sin_\lambda{\theta}\sin_\lambda^\ast{\phi}
    \end{align*}
\end{proposition}
\begin{proof}[Proof of \cref{M:Trigonometry:Sum}]
    Proof by exhaustion.
\end{proof}
\begin{proposition}[Derivative of generalized trigonometric functions]\label{M:Trigonometry:Derivative}
    \begin{align*}
        {\sin_\lambda}^\prime{\theta}      & =\lambda\cos_\lambda{\theta}         \\
        {\sin_\lambda^\ast}^\prime{\theta} & =\abs{\lambda}\cos_\lambda{\theta}   \\
        {\cos_\lambda}^\prime{\theta}      & =-\lambda\sin_\lambda^\ast{\theta}   \\
        {\tan_\lambda}^\prime{\theta}      & =\lambda\sec_\lambda^2{\theta}       \\
        {\tan_\lambda^\ast}^\prime{\theta} & =\abs{\lambda}\sec_\lambda^2{\theta}
    \end{align*}
\end{proposition}
\begin{proof}[Proof of \cref{M:Trigonometry:Derivative}]
    Proof by exhaustion.
\end{proof}
\subsubsection{Matrices}
\begin{definition}\label{M:Rotation}
    \textit{Generalized rotation matrix} is defined as
    \begin{align*}
        R_\lambda\left(\theta\right) & \coloneqq
        \begin{bmatrix}
            \cos_\lambda{\theta} & -\sin_\lambda^\ast{\theta} \\
            \sin_\lambda{\theta} & \cos_\lambda{\theta}       \\
        \end{bmatrix}\text{,}
    \end{align*}
    where \(\theta\in\mathbb{R}\).
\end{definition}
\begin{corollary}[Generalized rotation matrix at zero]\label{M:Rotation:Identity}
    \[
        R_\lambda\left(0\right)
        =
        I_2
    \]
\end{corollary}
\begin{proof}[Proof of \cref{M:Rotation:Identity}]
    Obvious
\end{proof}
\begin{corollary}[Generalized rotation matrix of sum of arguments]\label{M:Rotation:Sum}
    \[
        R_\lambda\left(\theta\right)R_\lambda\left(\phi\right)=R_\lambda\left(\theta+\phi\right)
    \]
\end{corollary}
\begin{proof}[Proof of \cref{M:Rotation:Sum}]
    \begin{align*}
        R_\lambda\left(\theta\right)R_\lambda\left(\phi\right)
         & =\begin{bmatrix}
                \cos_\lambda{\theta} & -\sin_\lambda^\ast{\theta} \\
                \sin_\lambda{\theta} & \cos_\lambda{\theta}       \\
            \end{bmatrix}
        \begin{bmatrix}
            \cos_\lambda{\phi} & -\sin_\lambda^\ast{\phi} \\
            \sin_\lambda{\phi} & \cos_\lambda{\phi}       \\
        \end{bmatrix}                                                                           \\
         & =\begin{bmatrix}
                \cos_\lambda{\theta}\cos_\lambda{\phi}+\left(-\sin_\lambda^\ast{\theta}\right)\sin_\lambda{\phi} &
                \cos_\lambda{\theta}\left(-\sin_\lambda^\ast{\phi}\right)+\left(-\sin_\lambda^\ast{\theta}\right)\cos_\lambda{\phi} \\
                \sin_\lambda{\theta}\cos_\lambda{\phi}+\cos_\lambda{\theta}\sin_\lambda{\phi}                    &
                \sin_\lambda{\theta}\left(-\sin_\lambda^\ast{\phi}\right)+\cos_\lambda{\theta}\cos_\lambda{\phi}                    \\
            \end{bmatrix} \\
         & =\begin{bmatrix}
                \cos_\lambda{\theta}\cos_\lambda{\phi}-\sin_\lambda^\ast{\theta}\sin_\lambda{\phi} &
                -\left(\sin_\lambda^\ast{\theta}\cos_\lambda{\phi}+\cos_\lambda{\theta}\sin_\lambda^\ast{\phi}\right) \\
                \sin_\lambda{\theta}\cos_\lambda{\phi}+\cos_\lambda{\theta}\sin_\lambda{\phi}      &
                \cos_\lambda{\theta}\cos_\lambda{\phi}-\sin_\lambda{\theta}\sin_\lambda^\ast{\phi}                    \\
            \end{bmatrix}               \\
         & =\begin{bmatrix}
                \cos_\lambda{\theta+\phi} & -\sin_\lambda^\ast{\theta+\phi} \\
                \sin_\lambda{\theta+\phi} & \cos_\lambda{\theta+\phi}       \\
            \end{bmatrix}                                                         \\
         & =R_\lambda\left(\theta+\phi\right)                                                                                   \\
        R_\lambda\left(\theta\right)
        R_\lambda\left(\phi\right)
         & =R_\lambda\left(\theta+\phi\right)
         & \qedhere
    \end{align*}
\end{proof}
\begin{corollary}[Inverse of generalized rotation matrix]\label{M:Rotation:Inverse}
    \[
        R_\lambda\left(\theta\right)^{-1}=R_\lambda\left(-\theta\right)
    \]
\end{corollary}
\begin{proof}[Proof of \cref{M:Rotation:Inverse}]
    \begin{align*}
        R_\lambda\left(\theta\right)
        R_\lambda\left(-\theta\right)
         & =R_\lambda\left(0\right) \\
         & =I_2                     \\
        R_\lambda\left(-\theta\right)
        R_\lambda\left(\theta\right)
         & =R_\lambda\left(0\right) \\
         & =I_2
    \end{align*}
    \[
        R_\lambda\left(\theta\right)R_\lambda\left(-\theta\right)
        =R_\lambda\left(-\theta\right)R_\lambda\left(\theta\right)
        =I_2
    \]
    \[
        {R_\lambda\left(\theta\right)}^{-1}
        =
        R_\lambda\left(-\theta\right)
        \qedhere
    \]
\end{proof}
\begin{definition}\label{M:Position}
    \textit{Position matrix} is defined recursively as
    \begin{align*}
        P_{\lambda,n}\left(\left\{\tensor{\theta}{^1},\dots,\tensor{\theta}{^n}\right\}\right)
                      & \coloneqq
        \begin{bmatrix}
            P_{\lambda,n-1}\left(\left\{\tensor{\theta}{^1},\dots,\tensor{\theta}{^{n-1}}\right\}\right) & 0_{n\times 1} \\
            0_{1\times n}                                                                                & 1             \\
        \end{bmatrix}
        T_{2,n+1}
        \begin{bmatrix}
            R_\lambda\left(\tensor{\theta}{^n}\right) & 0_{{2}\times{n-1}} \\
            0_{{n-1}\times{2}}                        & {I}_{n-1}          \\
        \end{bmatrix}
        T_{2,n+1}\text{,}                     \\
        P_{\lambda,0} & \coloneqq I_1\text{,}
    \end{align*}
    where \(\theta=\left\{\tensor{\theta}{^i}\right\}\in\mathbb{R}^n\) for \(i = 1, 2, \dots, n\).
\end{definition}
\begin{corollary}[Position matrix at zero]\label{M:Position:Identity}
    \[
        P_{\lambda,n}
        \left(0_{n}\right)
        =
        I_{n+1}
    \]
\end{corollary}
\begin{proof}[Proof of \cref{M:Position:Identity}]
    Prove by mathematical induction on \(n\),
    Let
    \begin{equation}\label{M:Position:Set:Identity:Proof:Induction}
        P_{\lambda,n-1}\left(0_{n-1}\right)=I_n
    \end{equation}
    \begin{align*}
        P_{\lambda, n}\left(0_{n}\right)
         & =
        \begin{bmatrix}
            P_{\lambda,n-1}\left(0_{n-1}\right) & 0_{n\times 1} \\
            0_{1\times n}                       & 1             \\
        \end{bmatrix}
        T_{2, n+1}
        \begin{bmatrix}
            R_{\lambda}\left(0\right) & 0_{{2}\times{n-1}} \\
            0_{{n-1}\times{2}}        & {I}_{n-1}          \\
        \end{bmatrix}
        T_{2, n+1} \\
         & =
        \begin{bmatrix}
            I_{n}         & 0_{n\times 1} \\
            0_{1\times n} & 1             \\
        \end{bmatrix}
        T_{2, n+1}
        \begin{bmatrix}
            R_{\lambda}\left(0\right) & 0_{{2}\times{n-1}} \\
            0_{{n-1}\times{2}}        & {I}_{n-1}          \\
        \end{bmatrix}
        T_{2, n+1} \\
         & =
        \begin{bmatrix}
            I_{n}         & 0_{n\times 1} \\
            0_{1\times n} & 1             \\
        \end{bmatrix}
        T_{2, n+1}
        \begin{bmatrix}
            I_{2}              & 0_{{2}\times{n-1}} \\
            0_{{n-1}\times{2}} & {I}_{n-1}          \\
        \end{bmatrix}
        T_{2, n+1} \\
         & =
        I_{n+1}
        T_{2, n+1}
        I_{n+1}
        T_{2, n+1} \\
         & =
        T_{2, n+1}
        T_{2, n+1} \\
         & =
        I_{n+1}
    \end{align*}
    \[
        P_{\lambda,n}
        \left(0_{n}\right)
        =
        I_{n+1}
        \qedhere
    \]
\end{proof}
\begin{definition}\label{M:Orientation}
    \textit{Orientation matrix} is defined as
    \begin{align*}
        Q^{\pm}_n\left(\phi_{n-1},\phi_{n-2},\dots,\phi_1\right)
         & \coloneqq
        \begin{bmatrix}
            1             & 0_{1\times n}                                                   \\
            0_{n\times 1} & X^{\pm}_{+1,n-1}\left(\phi_{n-1},\phi_{n-2},\dots,\phi_1\right) \\
        \end{bmatrix}\text{,} \\
        Q^{\pm}_0
         & \coloneqq\pm I_1\text{,}
    \end{align*}
    where \(\phi_m\in\mathbb{R}^m\) for \(m = 1, 2, \dots, n-1\).
\end{definition}
\begin{corollary}[Orientation matrix at zero]\label{M:Orientation:Identity}
    \[
        Q^{+}_{n}
        \left(0_{n-1}, 0_{n-2},\dots\right)
        =
        I_{n+1}
    \]
\end{corollary}
\begin{proof}[Proof of \cref{M:Orientation:Identity}]
    Prove by mathematical induction on \(n\),
    Let
    \begin{equation}\label{M:Orientation:Set:Identity:Proof:Induction}
        Q^{+}_{n-1}\left(0_{n-2}, 0_{n-3},\dots\right)=I_n
    \end{equation}
    \begin{align*}
        Q^{+}_n\left(0_{n-1}, 0_{n-2},\dots\right)
         & =
        \begin{bmatrix}
            1             & 0_{1\times n}                                      \\
            0_{n\times 1} & X^{\pm}_{+1,n-1}\left(0_{n-1},0_{n-2},\dots\right) \\
        \end{bmatrix}                           \\
         & =
        \begin{bmatrix}
            1             & 0_{1\times n}                                                                \\
            0_{n\times 1} & P_{+1,n-1}\left(0_{n-1}\right)Q^{+}_{n-1}\left(0_{n-2}, 0_{n-3},\dots\right) \\
        \end{bmatrix} \\
         & =
        \begin{bmatrix}
            1             & 0_{1\times n}                      \\
            0_{n\times 1} & P_{+1,n-1}\left(0_{n-1}\right) I_n \\
        \end{bmatrix}                                           \\
         & =
        \begin{bmatrix}
            1             & 0_{1\times n}                  \\
            0_{n\times 1} & P_{+1,n-1}\left(0_{n-1}\right) \\
        \end{bmatrix}                                               \\
         & =
        \begin{bmatrix}
            1             & 0_{1\times n} \\
            0_{n\times 1} & I_{n}         \\
        \end{bmatrix}                                                                \\
         & =
        I_{n+1}
    \end{align*}
    \[
        Q^{+}_n
        \left(0_{n-1}, 0_{n-2},\dots\right)
        =
        I_{n+1}
        \qedhere
    \]
\end{proof}
\begin{definition}\label{M:Point}
    \textit{Point matrix} is defined as
    \begin{align*}
        X^{\pm}_{\lambda,n}\left(\theta,\phi_{n-1},\phi_{n-2},\dots,\phi_1\right) & \coloneqq
        P_{\lambda,n}\left(\theta\right)
        Q^{\pm}_n\left(\phi_{n-1},\phi_{n-2},\dots,\phi_1\right)\text{,}
    \end{align*}
    where \(\theta\in\mathbb{R}^n\) and \(\phi_m\in\mathbb{R}^m\) for \(m = 1, 2, \dots, n-1\).
\end{definition}
\begin{corollary}[Position matrix as subset of point matrix]\label{M:Point:Position}
    \[
        P\left(\lambda,n\right)\subset X\left(\lambda,n\right)
    \]
\end{corollary}
\begin{proof}[Proof of \cref{M:Point:Position}]
    \begin{align*}
        \forall{P\in P\left(\lambda,n\right)}
        \forall{Q\in Q\left(n\right)},
         & P Q\in X\left(\lambda,n\right) \\
        \implies
         & P I\in X\left(\lambda,n\right) \\
        \implies
         & P\in X\left(\lambda,n\right)
    \end{align*}
    \[
        P\left(\lambda,n\right)\subset X\left(\lambda,n\right)\qedhere
    \]
\end{proof}
\begin{corollary}[Point matrix at zero]\label{M:Point:Identity}
    \[
        X^{+}_{\lambda,n}
        \left(0_{n}, 0_{n-1},\dots\right)
        =
        I_{n+1}
    \]
\end{corollary}
\begin{proof}[Proof of \cref{M:Point:Identity}]
    It can be implied from \cref{M:Position:Identity,M:Orientation:Identity}.
\end{proof}
\begin{corollary}[Closure of point matrix]\label{M:Point:Closure}
    A product of point matrices is a point matrix.
\end{corollary}
\begin{proof}[Proof of \cref{M:Point:Closure}]
    To be proven.
\end{proof}
\begin{corollary}[Inverse of point matrix]\label{M:Point:Inverse}
    A multiplicative inverse of point matrix is a point matrix.
\end{corollary}
\begin{proof}[Proof of \cref{M:Point:Inverse}]
    To be proven.
\end{proof}
\begin{remark}\label{M:Group}
    Since the set of point matrix together with matrix multiplication
    have closure (\cref{M:Point:Closure}),
    associativity (matrix multiplication),
    identity (\cref{M:Point:Identity}),
    and inverse (\cref{M:Point:Inverse}) property
    i.e. is a group in the algebraic sense according to \cref{Group},
    it is a valid candidate for the group \(\left(M,\otimes\right)\).
\end{remark}
\end{document}